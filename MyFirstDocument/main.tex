\documentclass{article}
\usepackage[utf8]{inputenc}
\usepackage[english,russian]{babel}
\title{Ускорение свободного падения}
\author{Игорь Байрамалов}
\date{\today}
\begin{document}
\section{Открытие гравитационной составляющей}
Сила притяжения Земли к Солнцу подчинялась формуле:
\begin{equation}
F=\frac {GmM}{b^2}
\end{equation}
Эксперименты показали, что множитель \(1/d^2\) в этом соотношении был вполне применим и в случае рассмотрения других планет в Солнечной системе. Постоянная G являлась коэффициентом, приводившим значение пропорции к числовой величине.

Руководствуясь собственной теорией, Ньютон измерил соотношения масс различных небесных тел, например масса Юпитера / масса Солнца, масса Луны / масса Земли, но численный ответ на вопрос о том, сколько весит Земля, Ньютон дать не мог, так как постоянная G по-прежнему оставалась неизвестной.

Величина гравитационной постоянной была открыта лишь спустя полвека после смерти Ньютона. Оценки этой величины на основе гипотез, подобных предположениям Ньютона, показали, что данная величина является ничтожно малой, и в земных условиях вычислить ее значение практически невозможно. Обычная сила тяжести кажется огромной, поскольку все знакомые нам предметы невообразимо малы по сравнению с массой земного шара.

\section{Измерение ускорения Земли}
Согласно третьему закону Ньютона сила притяжения двух тел зависит лишь от их массы и расстояния между ними. Таким образом, подставляя в правую часть уравнения множитель, известный из второго закона Ньютона, получаем:
\begin{equation}
ma=\frac {G(mM)}{b^2}
\end{equation}
В нашем случае массу m можно сократить, а величина а и есть ускорение, с которым тело m притягивается к Земле. В настоящее время ускорение свободного падения принято обозначать буквой g Получаем:
\begin{equation}
g=\frac {GM}{b^2}
\end{equation}
В нашем случае d –радиус Земли, М – ее масса, а G –та самая неуловимая константа, которую на протяжении многих лет искали физики. Подставляя в уравнение известные данные, получим: g=9,8м/\(c^2\). Эта величина и составляет ускорение свободного падения на Земле.
\section{Значения G для разных широт}
Поскольку наша планета не имеет форму шара, а является геоидом, радиус ее не везде одинаков. Земля как бы сплюснута, поэтому на экваторе и на обоих полюсах ускорение свободного падения будет принимать различные значения. В целом разница в показаниях длины радиуса составляет около 43 км. Поэтому в физике для решения задач принимается то ускорение свободного падения, которое измерено на широте около 450\(^0\) . Довольно часто для облегчения расчетов его принимают равным 10 м/\(c^2\).
\section{Значения G для Луны}
Наш спутник подчиняется тем же законам, что и остальные планеты Солнечной системы. Строго говоря, вычисляя ускорение на поверхности Луны, следует принимать во внимание и притяжение со стороны Солнца.
Но, как видно из формулы, с увеличением расстояния значение силы притяжения резко уменьшается. Поэтому, отбросив все второстепенные силы, используем ту же формулу:
\begin{equation}
ma=\frac {G(mM)}{b^2}
\end{equation}
Здесь М – масса Луны, а d – ее диаметр. Подставив известные величины, получим величину GЛ=1,622 м/\(c^2\). Эта величина и представляет собой ускорение свободного падения на Луне.

Именно такое малое значение GЛ является главной причиной того, что на Луне отсутствует атмосфера. По некоторым данным на заре времен наш спутник имел атмосферу, но из-за слабого притяжения Луна довольно быстро ее растеряла. Все планеты с большой массой обычно обладают собственной атмосферой. Ускорение свободного падения у них достаточно велико для того, чтобы не только не терять собственную атмосферу, но и прихватывать из космоса некоторое количество молекулярного газа. 

Подведем некоторые итоги. Ускорение свободного падения - это величина, которой обладает каждое материальное тело. Как ни удивительно это звучит, но все, что обладает массой, притягивает к себе окружающие предметы. Просто это притяжение настолько мало, что в обычной жизни не играет никакой роли. Тем не менее ученые серьезно относятся даже к самым маленьким физическим константам, ведь влияние, которое они оказывают на окружающий мир, до конца еще нами не изучено.
\end{document}
